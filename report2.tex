\documentclass{article}

\usepackage[legalpaper, margin=1in]{geometry}
\usepackage{amssymb}
\usepackage{amsmath}
\usepackage{algorithm}
\usepackage[noend]{algpseudocode}
\usepackage{tikz}
\usetikzlibrary{arrows,calc,shapes,backgrounds}
\usepackage{multirow}

\title{Heursitic Optimization Techniques\\186.814-2019W\\Programming exercise 2 report}
\date{\today}
\author{Marton Hajdu, 11849197}
\begin{document}
\maketitle
\section*{Getting started}
Relevant parts of the project have been rewritten in C++ in order to make it faster and more robust. Some issues regarding the rounding and integer representation were solved to get the same environment for calculations as in the Python project. Therefore only the genetic algorithm and the hybrid algorithm can be found in the project.\medskip\\
The new project is under the "c++" folder which also contains the results in "c++/results" subfolder. To run the project, on can run the script "run.sh" or manually call "cmake . \&\& make" from the root folder. The executable can be run with "out/tsptdmd [input] [result-file] [additional-result-file]" command. 
\section*{Genetic algorithm}
The solutions are encoded simply as permutations (this was already done so no actual encoding happens). The population consists of 5 specimens. An initial population of 10 specimens is generated from which only the better solutions are selected (the fitness is the objective value). The crossover operator selects randomly a less than 4 long sequence of vertices in both parents and orders them based on the other parent's sequence ordering. The mutation operator swaps two vertices in the tour randomly. The mutation is done to all specimens in the population. Overall, a 100 iterations (generations) are made. In the end a local search is done on the whole population.
\section*{Hybrid approach}
The high-level outline of the hybrid approach is the following:
\begin{algorithmic}[1]
	\State Create an initial solution with a randomized greedy construction heuristic
	\State If the solution is infeasible, try to make it feasible
	\State $i \gets 0$
	\While{$i < k$}
	\State Make a local search of the current solution
	\State If the solution is infeasible, try to make it feasible
	\If{current solution is better than best}
	\State $best \gets current$
	\EndIf
	\If{local search reached local optimum and it is 0}
	\State \textbf{break}
	\ElsIf{local search reached local optimum and it is not 0}
	\State generate random neighbor of current solution
	\State $i \gets i+1$
	\EndIf
	\EndWhile
\end{algorithmic}
The number of iterations was set to $k=100$.
\section*{Randomized greedy construction heuristic}
The construction starts with creating a priority queue of all edges, according to their closeness to $A/k$. We go through this list sequentially in iterations until we get a full tour. In each iteration, we start from the first element of the queue and try to add it. Then, we make a small number of steps forward in the queue randomly. The number of steps is maximized in a number $3$ but it can be changed to a higher number or set it to $1$ to make the heuristic deterministic. After an iteration, all edges are removed that are no longer valid.
\section*{Feasibility heuristic}
If the solution contains infeasible edges, this heuristic tries to get rid of those without worsening the solution. This is by iteratively doing a 2-opt exchange with edges so that the number of infeasible edges is not increasing. The already made exchanges are remembered and later ignored so that the heuristic does not fall into an infinite loop. After no changes are made, we return.
\section*{Local search}
A local search with best improvement is done first with reversal (2-opt) and then with pairwise driver exchange neighborhoods. The neighbors are not actually generated, only the best is created after evaluating all neighbors a priori.
\section*{Generating a random neighbor}
After getting stuck in a local optimum, the algorithm generates a neighbor of the current solution with the 2-opt exchange by consecutively applying it $k$ times. $k$ is set to 4 currently.
\section*{Setup for experiments}
The experimental setup that created the results used my home computer. The algorithm was set to terminate after reaching a 15 minute running time. Each heuristic was run with each input 10 times.
\section*{Results}
The columns show averages and standard deviation of the 10 runs for each instance. An additional measure of infeasible edges was added to measure the performance of instances where there could be infeasible solutions. All results can be found attached to the source code.
\small
\begin{center}
\begin{tabular}{|l|c|c|c|c|c|c|c|c|}
\hline
\multirow{3}{*}{Instance}& \multicolumn{4}{c}{Genetic algorithm} & \multicolumn{4}{c}{Hybrid} \\
& \multicolumn{2}{c}{solution} & \multicolumn{2}{c}{infeasible edges} & \multicolumn{2}{c}{solution} & \multicolumn{2}{c}{infeasible edges} \\
& avg & std.dev. & avg & std.dev. & avg & std.dev. & avg & std.dev. \\
\hline
0010\_k1 & 146.00 & 6.67 & 0 & 0 & 105.00 & 0 & 0 & 0\\
0010\_k2 & 297075.40 & 35353.99 & 0 & 0 & 205934.70 & 61.54 & 0 & 0\\
0015\_k1 & 384.30 & 27.54 & 0 & 0 & 325.00 & 0 & 0 & 0\\
0015\_k2 & 1330389.00 & 46169.62 & 0 & 0 & 1250989.00 & 3.43 & 0 & 0\\
0020\_k1 & 422.10 & 68.70 & 0 & 0 & 384.00 & 0 & 0 & 0\\
0020\_k2 & 1590635.60 & 275281.08 & 0 & 0 & 1111229.60 & 29976.46 & 0 & 0\\
0025\_k1 & 440407.30 & 1221303.23 & .10 & .30 & 521.00 & 0 & 0 & 0\\
0025\_k2 & 2150249.30 & 171936.23 & 0 & 0 & 1375280.30 & 46129.16 & 0 & 0\\
0030\_k1 & 1290342.10 & 1969916.98 & 0 & 0 & 655.50 & 29.78 & 0 & 0\\
0030\_k2 & 2640094.00 & 137465.49 & 0 & 0 & 1970314.60 & 143727.44 & 0 & 0\\
1000\_k1 & inf & 0 & 6.50 & 2.06 & 165758620.25 & 6972515.40 & .60 & 2.40\\
1000\_k2 & inf & 0 & 6.80 & 2.63 & 236753834 & 63090674 & 1.10 & 4.90\\
1500\_k1 & inf & 0 & 7.30 & 1.18 & inf & 0 & 1.50 & 2.50\\
1500\_k2 & inf & 0 & 9.40 & 2.83 & inf & 0 & 1.90 & 4.90\\
2000\_k1 & inf & 0 & 7.20 & 2.95 & inf & 0 & 2.80 & 5.60\\
2000\_k2 & inf & 0 & 10.20 & 3.37 & inf & 0 & 2.70 & 6.10\\
2500\_k1 & inf & 0 & 11.50 & 4.17 & inf & 0 & 2.90 & 12.90\\
2500\_k2 & inf & 0 & 12.00 & 4.95 & inf & 0 & 4.60 & 6.40\\
3000\_k1 & inf & 0 & 11.30 & 3.10 & inf & 0 & 4.00 & 10.00\\
3000\_k2 & inf & 0 & 9.90 & 2.07 & inf & 0 & 4.50 & 14.50\\
a280\_k1\_1 & 1312.70 & 139.42 & 0 & 0 & 293.30 & 47.42 & 0 & 0\\
a280\_k1\_2 & 0 & 0 & 0 & 0 & 0 & 0 & 0 & 0\\
a280\_k2\_1 & 670.70 & 26.27 & 0 & 0 & 114.10 & 27.64 & 0 & 0\\
a280\_k2\_2 & 12.50 & 13.23 & 0 & 0 & 6.60 & .48 & 0 & 0\\
a280\_k3\_1 & 422.90 & 21.04 & 0 & 0 & 70.10 & 14.57 & 0 & 0\\
a280\_k3\_2 & 10.00 & 3.00 & 0 & 0 & 8.50 & .50 & 0 & 0\\
a280\_k4\_1 & 331.60 & 16.36 & 0 & 0 & 44.90 & 11.83 & 0 & 0\\
a280\_k4\_2 & 7.70 & 2.64 & 0 & 0 & 6.10 & .30 & 0 & 0\\
a280\_k5\_1 & 278.00 & 14.59 & 0 & 0 & 42.90 & 11.21 & 0 & 0\\
a280\_k5\_2 & 11.50 & .92 & 0 & 0 & 10.80 & .60 & 0 & 0\\
berlin52\_k1\_1 & 938.40 & 195.05 & 0 & 0 & 205.40 & 126.49 & 0 & 0\\
berlin52\_k1\_2 & .30 & .45 & 0 & 0 & 0 & 0 & 0 & 0\\
berlin52\_k2\_1 & 461.60 & 41.25 & 0 & 0 & 72.00 & 52.62 & 0 & 0\\
berlin52\_k2\_2 & 4.90 & .70 & 0 & 0 & 4.00 & 0 & 0 & 0\\
berlin52\_k3\_1 & 376.40 & 74.24 & 0 & 0 & 45.10 & 24.45 & 0 & 0\\
berlin52\_k3\_2 & 4.10 & .30 & 0 & 0 & 4.10 & .53 & 0 & 0\\
berlin52\_k4\_1 & 244.70 & 41.66 & 0 & 0 & 42.30 & 21.93 & 0 & 0\\
berlin52\_k4\_2 & 4.50 & .50 & 0 & 0 & 4.70 & .45 & 0 & 0\\
berlin52\_k5\_1 & 251.00 & 45.60 & 0 & 0 & 32.30 & 4.07 & 0 & 0\\
berlin52\_k5\_2 & 5.90 & .70 & 0 & 0 & 5.10 & .53 & 0 & 0\\
rl5915\_k1\_1 & 655498.20 & 8191.88 & 0 & 0 & 560280.40 & 14826.84 & 0 & 0\\
rl5915\_k1\_2 & 355051.00 & 7252.20 & 0 & 0 & 266141.60 & 12141.47 & 0 & 0\\
rl5915\_k2\_1 & 328042.20 & 3612.17 & 0 & 0 & 277239.00 & 3186.52 & 0 & 0\\
rl5915\_k2\_2 & 179491.10 & 4378.46 & 0 & 0 & 138010.60 & 5150.40 & 0 & 0\\

\hline
\end{tabular}
\end{center}
\end{document}