\documentclass{article}

\usepackage{amssymb}
\usepackage{amsmath}
\usepackage{algorithm}
\usepackage[noend]{algpseudocode}
\usepackage{tikz}
\usetikzlibrary{arrows,calc,shapes,backgrounds}

\title{Heursitic Optimization Techniques\\186.814-2019W\\Programming exercise 1 report}
\date{\today}
\author{Marton Hajdu,\\matriculation number: 11849197}
\begin{document}
\maketitle
\section*{Subtask 1}
We get a relatively bad value if we select the largest weight edges and assign all of them to a single driver:
$$\sqrt{\frac{1}{k}(L-\sum_{e\in\phi^{-1}(c_i)}{d_e})^2+\frac{1}{k}\sum_{i=2}^{k}{L^2}}$$
The question is whether it is the worst case. Under the assumption that $L$ not close to the sum of $n$ maximum value edges selected for the tour, the first component of the square root decreases if we change any edge to a lower weight edge, and if we assign some edge to another driver, the first component decreases more than the second component can possibly increase.\medskip\\
This means that selecting the first $n$ maximum weight edges and assigning them to a single driver gives an upper bound for the objective and any value greater than this $n$ maximum weight sum is a good value for $M$.
\section*{Subtask 2}
One simple deterministic construction heuristic is to pick $\frac{n}{k}$ edges for all $k$ drivers. We calculate the average of the to-be-added edge weights $A$ to get $\frac{n}{k}\cdot A$ close to $L$. We try to pick edges that have weight close to $A$ and which make no cycle with the edges already added. We assign these edges to drivers in a cyclic manner:
\begin{algorithmic}[1]
\Function{DeterministicConstruction}{E, n, k, L}
\State $A \gets L*k/n$
\State find index $i$ in $E$ with weight closest to $A$
\State $j \gets i+1, l \gets 0$
\While{less than $n$ edges are added}
\If{$i \ge 0$ and $E[i]$ does not form a cycle with previous edges}
\State add $E[i]$ to edge set and to $l$th driver
\State $l \gets l + 1 \mod k$
\EndIf
\If{$j < m$ and $E[j]$ does not form a cycle with previous edges}
\State add $E[j]$ to edge set and to $l$th driver
\State $l \gets l + 1 \mod k$
\EndIf
\State $i \gets i - 1, j \gets j + 1$
\EndWhile
\EndFunction
\end{algorithmic} 
\section*{Solution representation}
The solution is represented by a circular permutation of the edges of the tour with a pointer from each edge to the driver associated with it. Also, we have the values for the tour and the driver stored as described in the next section.\medskip\\
A tricky part of the solution representation is to maintain a valid partial solution during the construction part. Then when we insert a new edge, one of the following happens:
\begin{itemize}
	\item It forms a completely new chain, i.e. it's endpoints are not present in the current partial solution. In this case, we simply add this edge to the end:
	$$(1,3),(3,6),(5,4),(4,8) \rightarrow (1,3),(3,6),(5,4),(4,8),{\color{red}(2,7)}$$
	\item It connects to one chain of edges, in this case we need to connect to one of the two ends of this chain (inserting $(2,1)$ in the first example and $(2,6)$ in the second):
	$$(1,3),(3,6),(5,4),(4,8) \rightarrow {\color{red}(2,1)},(1,3),(3,6),(5,4),(4,8)$$
	$$(1,3),(3,6),(5,4),(4,8) \rightarrow (1,3),(3,6),{\color{red}(6,2)},(5,4),(4,8)$$
	\item It connect to two chains of edges. If the chains are the same, we don't insert the edge as it creates an invalid solution with a cycle of less than $n$ vertices. Otherwise we may need to connect the two chains, with possibly flipping one entirely, edge by edge:
	$$(1,3),(3,6),(5,4),(4,8) \rightarrow (1,3),(3,6),{\color{red}(6,8),(8,4),(4,5)}$$
\end{itemize}

\section*{Delta evaluation}
We store the following values:
\begin{itemize}
	\item the objective value $F$ without taking its square root
	\item each value $S_i := \frac{1}{k}(L-\sum_{e\in\phi^{-1}(c_i)}{d_e})^2$ for the driver $i$ so that we can simply remove the squared value when something changes
	\item the inner $D_i := L-\sum_{e\in\phi^{-1}(c_i)}$ for the $i$th driver so that we can create the updated squared value after a change
\end{itemize}
When the solution changes, one of the following happens:
\begin{itemize}
	\item remove an edge from a driver: $F = F - S_i$, then we remove the edge weight from $D_i$, $S_i = D_i^2$ and $F = F + S_i$.
	\item assign an edge to a driver: $F = F - S_i$, then we add the edge weight to $D_i$, $S_i = D_i^2$ and $F = F + S_i$
	\item change an edge of a driver to another edge: $F = F - S_i$, then we add the difference $w(e_{new}) - w(e_{old})$ of the edge weights to $D_i$, $S_i = D_i^2$ and $F = F + S_i$
	\item chain of the above: when we change multiple edges at a driver, we only calculate $F$ and $S_i$ when all are added or when we change multiple driver assignments, we only calculate $F$ at the end
\end{itemize}
\end{document}